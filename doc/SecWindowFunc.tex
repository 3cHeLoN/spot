\chapter{Window functions}\label{Sec:SparcoWindowFunc}

% Parzen window unclear, scale discrete mesh from -1 to 1?

% On the use of windows for harmonic analysis with the discrete Fourier transform
% Harris, F.J.  
% This paper appears in: Proceedings of the IEEE
% Publication Date: Jan. 1978
% Volume: 66,  Issue: 1
% On page(s): 51- 83
% ISSN: 0018-9219
% Date Published in Issue: 2005-06-28 14:49:59.0
% Abstract: This paper makes available a concise review of data
% windows and their affect on the detection of harmonic signals in the
% presence of broad-band noise, and in the presence of nearby strong
% harmonic interference. We also call attention to a number of common
% errors in the application of windows when used with the fast Fourier
% transform. This paper includes a comprehensive catalog of data
% windows along with their significant performance parameters from
% which the different windows can be compared. Finally, an example
% demonstrates the use and value of windows to resolve closely spaced
% harmonic signals characterized by large differences in amplitude.

% N is number of samples, n  http://en.wikipedia.org/wiki/Window_function

Define $k:= \left\vert \frac{2n}{N-1} - 1\right\vert$, for
$n=0,\ldots,N-1$, as the normalized offset from the center; i.e.,
corresponding to the samples $0 \leq \vert \ell\vert \leq
\frac{N}{2}$, normalized to the range $0$ to $1$.

\begin{longtable}{|p{3.1cm}p{10cm}|}
\hline
Bartlett & 
      $\displaystyle w(k) = 1 - k$ \\
&\\ Bartlett-Hann &
   $\displaystyle w(n) = a_0 -
   a_1\left\vert \frac{n}{N-1}- \frac{1}{2}\right\vert -
   a_2\cos\left(\frac{2\pi n}{N-1}\right)$ \\ 
  & \footnotesize with $a_0 = 0.62$, $a_1 = 0.48$, $a_2 = 0.38$ \\
&\\ Blackman &
   $\displaystyle w_{\alpha}(n) = \sum_{j=0}^2 a_j(-1)^j \cos\left(\frac{2j\pi
   n}{N-1}\right)$ \\
  & \footnotesize with $a_0 = (1-\alpha)/2$, $a_1 = 1/2$, $a_2 = \alpha / 2$ \\
&\\ Blackman-Harris &
   $\displaystyle w(n) = \sum_{j=0}^3 a_j(-1)^j \cos\left(\frac{2j\pi
   n}{N-1}\right)$ \\
  & \footnotesize with $a_0 = 0.35875$, $a_1 = 0.48829$, $a_2 = 0.14128$, $a_3 = 0.01168$ \\
&\\ Blackman-Nuttall &
   $\displaystyle w(n) = \sum_{j=0}^3 a_j(-1)^j \cos\left(\frac{2j\pi
   n}{N-1}\right)$ \\
  & \footnotesize with $a_0 = 0.3635819$, $a_1 = 0.4891775$,
   $a_2 = 0.1365995$, $a_3 = 0.0106411$ \\
&\\ Bohman &
   $\displaystyle w(k) = (1 - k)
   \cos(\pi k) +
   \frac{1}{\pi}\sin(\pi k) $ \\
&\\ Cauchy &
   $\displaystyle w_{\alpha}(k) = \frac{1}{1 + (\alpha\cdot k)^2}$ \\
&\\ Cos & See cosine \\
&\\ Cosine &
   $\displaystyle w_{\alpha}(k) = cos^{\alpha}\left(\frac{k\pi}{2}\right)$ \\
&\\ Dirichlet & 
   $\displaystyle w(k) = 1$ \\
&\\ Flat top / Flattop&
   $\displaystyle w_{\alpha}(n) = \textstyle\frac{1}{\sum_{j=0}^4 a_i}
   \displaystyle \sum_{j=0}^4 a_j(-1)^j \cos\left(\frac{2j\pi
   n}{N-1}\right)$ \\
  & \footnotesize with $a_0 = 1$, $a_1 = 1.93$, $a_2 = 1.29$, $a_3 =
  0.388$, $a_4 = 0.032$ \\
&\\ Gauss & See Gaussian \\
&\\ Gaussian & 
  $\displaystyle w_{\alpha}(k) =
  \mathrm{exp}\left(-\frac{1}{2}\left(\alpha k\right)^2\right)$\\
&\\ Hamming &
   $\displaystyle w(n) = a_0 - a_1\cos\left(\frac{2\pi
   n}{N-1}\right)$  \\
   & \footnotesize with $a_0 = 0.54$, $a_1 = 0.46$ \\
&\\ Hann &
  $\displaystyle w(n) = \frac{1}{2}\left(1 - \cos\left(\frac{2\pi
  n}{N-1}\right)\right)$ \\
&\\ Hann-Poisson &
  $\displaystyle w_{\alpha}(k) = \frac{1}{2}\left[1 + \cos\left(\pi k)\right]\right)\exp\left(-ak\right)$ \\
&\\ Kaiser &
   $\displaystyle w_{\alpha}(k) = I_0\left(\alpha\sqrt{1 -
   k^2}\right)/I_0(\alpha)$  \\
   & \footnotesize with $I_0$ the zeroth-order modified Bessel
   function of the first kind \\
&\\ Kaiser-Bessel\ \ \ \ \ \ \ \ \ Derived (KBD) & 
   $\displaystyle w_{\alpha}(n) = \begin{cases} 
   c\left(\sum_{j=0}^n v_{\alpha}(j)\right)^{1/2},
   & \ \ 0 \leq n < N/2 \\
   c\left(\sum_{j=0}^{N-n-1} v_{\alpha}(j)\right)^{1/2},
   
   & \ \ N/2 \leq n < N
 \end{cases}$ \\
 & \footnotesize with $c=\left(\sum_{j=0}^{N/2}
   v_{\alpha}(j)\right)^{-1/2}$, $v_{\alpha}(n)$ the $(N/2+1)$-point
 Kaiser
 window. $N$ must be an even number.\\
   % http://ccrma.stanford.edu/courses/422/projects/kbd/kbdwindow.cpp
   % http://en.wikipedia.org/wiki/Kaiser_window
&\\ Lanczos &
   $\displaystyle w_{\alpha}(k) = \left(\frac{\sin(\pi k)}{\pi k}\right)^{\alpha}$ \\
&\\ Nuttall &
   $\displaystyle w(n) = \sum_{j=0}^3 a_j(-1)^j \cos\left(\frac{2j\pi
   n}{N-1}\right)$ \\
  & \footnotesize with $a_0 = 0.355768$, $a_1 = 0.487396$,
   $a_2 = 0.144232$, $a_3 = 0.012604$ \\
&\\ Parzen & 
   $\displaystyle w(t) = \begin{cases}
   1 - 6\left(\frac{t}{N/2}\right)^2\left(1 - \frac{\vert
       t\vert}{N/2}\right), & \ \ 0 \leq \vert t\vert \leq \frac{N}{4} \\
   2\left(1 - \frac{\vert t\vert}{N/2}\right)^3, &
   \ \ \frac{N}{4} \leq \vert t\vert \leq \frac{N}{2}   
   \end{cases}$ \\
% Mathematical Considerations in the Estimation of Spectra
% Emanuel Parzen
% Technometrics, Vol. 3, No. 2 (May, 1961), pp. 167-190 
&\\ Poisson & See Lanczos, $\alpha = 1$
   $\displaystyle w(k) = \exp\left(-ak\right)$ \\
&\\ Rectangle & See Dirichlet \\
&\\ Riesz &
   $\displaystyle w(k) = 1 - \vert k\vert^2$ \\
&\\ Sinc & See Lanczos \\
&\\ Triangle & 
   $\displaystyle w(k) = 1 - k$\\
&\\ Tukey & 
   $\displaystyle w_{\alpha}(t) = \begin{cases}
   1, & 0 \leq \vert t\vert \leq \alpha\frac{N}{2} \\
   \displaystyle\frac{1}{2} + \frac{1}{2}\cos\left(\frac{\pi}{2}\cdot
   \frac{2t/N - \alpha}{1 - \alpha}\right), &
   \alpha\frac{N}{2} \leq \vert
   t\vert \leq \frac{N}{2}
   \end{cases}$ \\
&\\ Uniform & See Dirichlet \\
&\\ Vall\'e-Poussin & See Parzen \\
&\\ Weierstrass & See Gaussian \\
\hline
\end{longtable}

%%% Local Variables: 
%%% mode: latex
%%% TeX-master: "manual"
%%% End: 
